\begin{table}[!h]
\centering\centering
\caption{Overtime Worker and Working Hour Distribution by Cluster}
\centering
\fontsize{11}{13}\selectfont
\begin{threeparttable}
\begin{tabular}[t]{lccc}
\toprule
List of clusters & Non-Overtime Worker & Overtime Worker & Total\\
\midrule
Cluster 1 & 37.054 & 51.031 & 39.925\\
 & {}[7.571] & {}[10.662] & {}[10.039]\\
 & 6072 & 1570 & 7642\\
Cluster 2 & 37.513 & 50.478 & 41.803\\
 & {}[6.382] & {}[7.998] & {}[9.254]\\
\addlinespace
 & 16213 & 8017 & 24230\\
Cluster 3 & 39.418 & 49.205 & 42.397\\
 & {}[3.112] & {}[6.631] & {}[6.356]\\
 & 10624 & 4649 & 15273\\
Cluster 4 & 36.515 & 56.461 & 46.633\\
\addlinespace
 & {}[7.364] & {}[10.683] & {}[13.567]\\
 & 2598 & 2675 & 5273\\
\hline\noalign{\vskip -0.1ex}\\
Total & 37.932 & 51.126 & 42.188\\
 & {}[6.023] & {}[8.759] & {}[9.347]\\
 & 35507 & 16911 & 52418\\
\bottomrule
\end{tabular}
\begin{tablenotes}
\item[1] Standard deviations in brakets. The number below each cell indicates the frequency (sample size) of the respective subgroup.
\end{tablenotes}
\end{threeparttable}
\end{table}
